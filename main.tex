\documentclass{article}
\usepackage[utf8]{inputenc}

%\title{Fine-tuning on Simulated Data to Overcome the Gap with Observed data}
\title{Overcome the Gap between Simulated and Observed Data in Industry 4.0\\ using Fine Tuning methods}

%\author{}
%\date{September 2019}

\begin{document}

\maketitle

\begin{abstract}
To become later ..
\end{abstract}

\section{Introduction}

\subsection{Industry and Mechanic}
In this part we would present (@Dmitry: these are just our suggestions, you can write this part better than us):
\begin{itemize}
\item A general paragraph on the industrial context such as the difficulty of generating real data for the industrial section
\item What is the necessity of using simulators (such as NCSIMUL) for the industrial world. 
\item What is he interest/difficulty of adapting the simulated data to the real observed data.
\item A general description on UF1 protocol (May be we should put this part in section 4.1. I am not sure.)
\item related works from the industrial point of view
\end{itemize}
 
\subsection{Machine Learning}
\begin{itemize}
	\item Explain how existed machine learning methods are effective for reducing the gap between simulated and observed data.
    \item related works on neural networks, fine tuning and domain adaptation.
    \item May be some related works on few shot learning. 
    \item if there exists any, related works on the successful works on reducing the gab between  simulated and real data.
\end{itemize}
\section{Context}

\section{Proposed approach}
\cite{•}
\subsection{Curves Alignment}
In this section, explain our approach (data preparation + dynamic time wrapping) for aligning two curves and motivation of introducing this measure as a loss function namely $l$(curve1, curve2). 
\subsection{Classical ML methods on adaptive learning}
\subsection{Various neural networks + fine tuning}
\subsection{Learning on simulated Data and adaptation on real Data}
\section{Experimental Setup}
\subsection{GP2R Presentation}
@Dmitry You can present the GP2R here or in another part of the paper if you have a suggestion.
\subsection{Dataset}
In this section we will present the data from Machine Learning point of view. And our setting for neural networks and other machine learning methods.
\begin{itemize}
\item explain all used parameters such as $F, S, A_e, A_p$ and power 
\item Real data set including materials, tools and the power definition of this dataset
\item COM data and its preparation 
\item NCSIMUL data and its power definition
\end{itemize} 
\subsection{Experimental Results}
To be completed by Pegah. \\
We follow a general approach for each pair of real vs ncsimul in our experiments:
\begin{itemize}
\item[1] Alignment between real and ncsimul curves
\item[2] Train an ML method ($F$) on COM and test on real and measure the lost i.e. $l$(real, F(real)).
\item[3] Train a model ($G$) on NCSIMUL + COM and check $l$(real, G(real))
\item[4] Train a model ($H$) on NCSIMUL + COM + Real and check $l$(real, H(real)).
\end{itemize}
What we expect to get as experimental results is to get a better loss (l value) in this order: $2 < 3 < 4$. 
\section{Conclusion and future works}

\end{document}
